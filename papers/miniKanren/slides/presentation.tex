\documentclass[xcolor=table]{beamer}
\usepackage{beamerthemesplit}
\usepackage{wrapfig}
\usetheme{SPbGU}
\usepackage{pdfpages}
\usepackage{amsmath}
\usepackage{amssymb}
\usepackage{cmap}
\usepackage[T2A]{fontenc}
\usepackage[utf8]{inputenc}
\usepackage[english]{babel}
\usepackage{indentfirst}
\usepackage{tikz}
\usetikzlibrary{shapes,arrows,automata,positioning,quotes,backgrounds,decorations.text,decorations.pathmorphing}
\usepackage{multirow}
\usepackage[noend]{algpseudocode}
\usepackage{algorithm}
\usepackage{algorithmicx}
\usepackage{fancyvrb}
\usepackage[linguistics]{forest}
\usepackage{listings}
\usepackage{multicol}
\usepackage{comment}
\usepackage{xspace}
\usepackage{adjustbox}
\usepackage{makecell}

\setbeamertemplate{itemize items}[circle]
\setbeamertemplate{enumerate items}[circle]

\lstdefinelanguage{ocanren}{
keywords={run, conde, fresh, let, in, match, with, when, class, type,
object, method, of, rec, repeat, until, while, \begin{comment}not,\end{comment} do, done, as, val, inherit,
new, module, sig, deriving, datatype, struct, if, then, else, open, private, virtual, include, success, failure,
true, false},
sensitive=true,
commentstyle=\small\itshape\ttfamily,
keywordstyle=\textbf,%\ttfamily\underline,
identifierstyle=\ttfamily,
basewidth={0.5em,0.5em},
columns=fixed,
mathescape=true,
fontadjust=true,
literate={fun}{{$\lambda$}}1 {function}{function}8 {->}{{$\to$}}3 {===}{{$\equiv$}}1 {=/=}{{$\not\equiv$}}1 {|>}{{$\triangleright$}}3 {\\/}{{$\vee$}}2 {/\\}{{$\wedge$}}2 {^}{{$\uparrow$}}1,
morecomment=[s]{(*}{*)},
 moredelim=**[is][\color{red}]{@!}{@}
}

\tikzstyle{processTree} = [
  ->,
  sibling distance=15em,
  scale=0.6,
  every node/.style = {
    shape=rectangle,
    rounded corners=0.05cm,
    draw,
    align=center,
    minimum size=5mm,
    scale=0.6,},
  %level 1/.style={sibling distance=100em}
  ]


\tikzstyle{program} = [
  draw=black,
  thick,
  rectangle,
  rounded corners=1pt,
  inner sep=5pt,
  inner ysep=5pt
  ]

\tikzstyle{goal} = [
  draw=black,
  rectangle,
  rounded corners=1pt,
  inner ysep=0pt,
  ]

\tikzstyle{input} = [
  draw=none,
  rectangle,
  rounded corners=1pt,
  inner sep=2pt,
  inner ysep=2pt,
  fill=green!10,
  minimum height=5mm
  ]


\tikzstyle{transparent} = [
  draw=none,
  inner ysep=3pt
  ]

\lstset{
language=ocanren
}

\newcommand{\mk}{\textsc{miniKanren}\xspace}
\renewcommand{\and}{$\&$\xspace}
\newcommand{\rel}[2]{\texttt{#1}$^o$ #2}
\newcommand{\subst}[1]{$\langle$#1$\rangle$}

\beamertemplatenavigationsymbolsempty

\title[Partial Deduction for \mk{}]{An Empirical Study of Partial Deduction for \mk{}}
\institute[JetBrains Research]{
JetBrains Research, Programming Languages and Tools Lab  \\
Saint Petersburg State University
}

\author[Kate Verbitskaia]{\textbf{Kate Verbitskaia}, Daniil Berezun, Dmitry Boulytchev}

\date{27.08.2020}

\definecolor{orange}{RGB}{179,36,31}

\begin{document}
{
\begin{frame}[fragile]
  \begin{tabular}{p{5.5cm} p{5.5cm}}
   \begin{center}
      \includegraphics[height=1.5cm]{pictures/jetbrainsResearch.pdf}
    \end{center}
    &
    \begin{center}
      \includegraphics[height=1.5cm]{pictures/SPbGU_Logo.png}
    \end{center}
  \end{tabular}
  \titlepage
\end{frame}
}

\begin{frame}[fragile]
  \frametitle{Specialization: a Method to Improve Programs}
\begin{center}
  \begin{tikzpicture}[
  node distance = 10mm and 11 mm,
  decoration = {
    snake,
    pre length=2pt,
    post length=2pt
  },
  remember picture,
  overlay]
  \node (a) [
    program,
    anchor=north west,
    xshift=0.4cm,
    yshift=-1.7cm]
  at (current page.north west)
  {
    \adjustbox{scale=0.7}{
      \begin{minipage}[c]{0.63\textwidth}
        \begin{lstlisting}
let rec eval$^o$ fm s r =
  fm === neg x & not$^o$ a r & eval$^o$ x s a |
  ...
        \end{lstlisting}
      \end{minipage}
    }
  };

  \node [transparent,anchor=south] at (a.north) {
    \footnotesize
    input program
  };

  \pause

  \node (c) [goal,anchor=north east] at (a.south east)
  {
    \adjustbox{scale=0.7}{
      \begin{minipage}[c]{0.25\textwidth}
        \begin{lstlisting}
eval$^o$ fm s true
        \end{lstlisting}
      \end{minipage}
    }
  };

  \node (lbl) [transparent,anchor=south west] at (a.east) {
    \footnotesize
    known argument
  };

  \draw [dashed,->] (lbl.south) to [out=270,in=0] ($(c.east)-(0.2,0)$);
  \pause

  \node (d) [input, below=of a]
  {
    \adjustbox{scale=0.7}{
      \begin{minipage}[c]{0.65\textwidth}
        \begin{lstlisting}
fm === neg x & not$^o$ a true & eval$^o$ x s a |
...
        \end{lstlisting}
      \end{minipage}
    }
  };

  \path[draw=black,->,thick, decorate] (a.south) to (d.north);

  \pause

  \node (e) [input, below=of d]
  {
    \adjustbox{scale=0.7}{
      \begin{minipage}[c]{0.5\textwidth}
        \begin{lstlisting}
fm === neg x & eval$^o$ x s false |
...
        \end{lstlisting}
      \end{minipage}
    }
  };

  \path[draw=black,->,thick, decorate] (d.south) to (e.north);

  \pause

  \node (dots) [input, below=of e] {...};

  \path[draw=black,->,thick, decorate] (e.south) to (dots.north);

  \pause

  \node (b) [
    program,
    anchor=south east,
    xshift=-0.4cm,
    yshift=0.7cm]
  at (current page.south east)
  {
    \adjustbox{scale=0.7}{
      \begin{minipage}[c]{0.53\textwidth}
        \begin{lstlisting}
let rec eval_true$^o$ fm s =
  fm === neg x & eval_false$^o$ x s |
  ...

let rec eval_false$^o$ fm s =
  fm === neg x & eval_true$^o$ x s |
  ...
        \end{lstlisting}
      \end{minipage}
    }
  };

  \node [transparent,anchor=south] at (b.north) {
    \footnotesize
    output
  };
\onslide<1->
\end{tikzpicture}

\end{center}
\end{frame}

\begin{frame}[fragile]
  \frametitle{Partial Deduction: Specialization for Logic Programming}
\begin{center}
  \begin{tikzpicture}[
  node distance = 10mm and 18 mm,
  decoration = {
    snake,
    pre length=2pt,
    post length=2pt
  }]
  \node (a) [program]
  {
    \adjustbox{scale=0.7}{
      \begin{minipage}[c]{0.47\textwidth}
        \begin{lstlisting}
let double_append$^o$ x y z r =
  ocanren {
    fresh t in
      append$^o$ x y t &
      append$^o$ t z r}
\end{lstlisting}

        \begin{lstlisting}
let rec append$^o$ x y r =
  ocanren {
    (x === [] & y === r) |
    (fresh h x' r' in
      x === h :: x' &
      append$^o$ x' y r' &
      r === h :: r')}
\end{lstlisting}
      \end{minipage}
    }
  };

  \node (b) [goal,anchor=north east] at (a.south east)
  {
    \adjustbox{scale=0.7}{
      \begin{minipage}[c]{0.35\textwidth}
        \begin{lstlisting}
double_append$^o$ x y z r
        \end{lstlisting}
      \end{minipage}
    }
  };

  \node [transparent, anchor=south] at (a.north) {\footnotesize input};


  \node (c) [program, right=of a.north east, anchor=north west]
  {
    \adjustbox{scale=0.7}{
      \begin{minipage}[c]{0.52\textwidth}
        \begin{lstlisting}
let double_append$^o$ x y z r =
  ocanren {
    (x === [] & append$^o$ y z r) |
    (fresh h x' r' in
      x === h :: x' &
      double_append$^o$ x' y z r' &
      r === h :: r')}
\end{lstlisting}

        \begin{lstlisting}
let rec append$^o$ x y r =
  ocanren {
    (x === [] & y === r) |
    (fresh h x' r' in
      x === h :: x' &
      append$^o$ x' y r' &
      r === h :: r')}
\end{lstlisting}
      \end{minipage}
    }
  };

  \node [transparent, anchor=south] at (c.north) {\footnotesize output};


\end{tikzpicture}
\end{center}
\end{frame}

\begin{frame}[fragile]
  \frametitle{Partial Deduction for \mk: Bird's-eye View}
  \begin{center}
\begin{tikzpicture}[
  node distance = 14mm and 13 mm,
  decoration = {
    snake,
    pre length=2pt,
    post length=4pt,
    amplitude=0.5pt,
    segment length=4pt
  },
  remember picture,overlay]
  \node (a) [
    program,
    anchor=north west,
    xshift=1cm,
    yshift=-1.8cm]
  at (current page.north west)
  {
    \adjustbox{scale=0.65}{
      \begin{minipage}[c]{0.38\textwidth}
        \begin{lstlisting}
let rec eval$^o$ fm s r =
  ...
  fm === conj x y &
  and$^o$ a b r &
  eval$^o$ x s a &
  eval$^o$ y s b |
  ...
        \end{lstlisting}
      \end{minipage}
    }
  };

  \node (b) [goal,anchor=north east] at (a.south east)
  {
    \adjustbox{scale=0.65}{
      \begin{minipage}[c]{0.24\textwidth}
        \begin{lstlisting}
eval$^o$ fm s true
        \end{lstlisting}
      \end{minipage}
    }
  };

  \pause

  \node (d) [
    transparent,
    xshift=-1cm,
    yshift=-1.8cm,
    anchor=north east]
  at (current page.north east)
  {
      \begin{tikzpicture}[
  processTree,
  sibling distance=7em,
  level distance=7em,
  level 2/.style={level distance=5em},
  anchor=center]
  \node {
    \rel{eval}{$fm \ s \ true$}}
    child { node[draw=none, fill=none] {...}}
    child { node {
      \rel{and}{$a \ b \ true$} \and  \\
      \rel{eval}{$x \ s \ a$} \and \\
      \rel{eval}{$y \ s \ b$} \\
      \subst{$fm \to conj \ x$}}
      child { node[draw=none, fill=none] {...}}
      }
    child { node[draw=none, fill=none] {...}}
  ;
\end{tikzpicture}
  };

  \draw[->,semithick, decorate]
    (a.north east) to
    [out=15,in=165,"{\footnotesize driving}"]
    ($(d.north west)+(0.9,0)$);

  \pause

  \node (e) [transparent, anchor=north east] at ($(d.south east)+(0,-1)$)
  {
    \begin{minipage}[c]{0.4\textwidth}
      \begin{tikzpicture}[
  processTree,
  sibling distance=4em,
  level distance=3em,
  level 3/.style={level distance=4em,sibling distance=8em},
  anchor=center]
  \node (root) {
    \rel{eval}{$fm \ s \ true$}}
    child { node[draw=none, fill=none] {...}}
    child { node[draw=none, fill=none] {...}
      child { node {
        \rel{eval}{$x \ s \ true$} \and \\
        \rel{eval}{$y \ s \ true$}}
        child { node (1) {\rel{eval}{$x \ s \ true$}}}
        child { node (2) {\rel{eval}{$y \ s \ true$}}}
        }
    }
    child { node[draw=none, fill=none] {...}}
  ;

  \draw [dashed,->] (1.west) to [out=170,in=-150] (root.west);
  \draw [dashed,->] (2.east) to [out=10,in=-30] (root.east);


\end{tikzpicture}
    \end{minipage}
  };

  \draw[->,semithick, decorate]
    ($(d.south)+(0.6,0.7)$) to
    [out=-75,in=45,"{\footnotesize folding}"]
    (e.north);

  \pause


  \node (c) [
    program,
    anchor=south west,
    xshift=1cm,
    yshift=1cm]
  at (current page.south west)
  {
    \adjustbox{scale=0.65}{
      \begin{minipage}[c]{0.4\textwidth}
        \begin{lstlisting}
let rec eval$^o$_true fm s =
  ...
  eval$^o$_true x s &
  eval$^o$_true y s |
  ...
        \end{lstlisting}
      \end{minipage}
    }
  };

  \draw[->,semithick, decorate]
    ($(e.west)+(0.45,0)$) to
    [out=200,in=-35,"{\footnotesize residualization}",pos=0.4]
    (c.east);

    \onslide<1->
\end{tikzpicture}
  \end{center}
\end{frame}

\begin{frame}[fragile]
  \frametitle{Driving: Unfolding}
  \begin{center}
    

\begin{tikzpicture}[
  decoration = {
    snake,
    pre length=2pt,
    post length=4pt,
    amplitude=0.5pt,
    segment length=4pt,
  },
  remember picture,overlay]
  % \draw[thick] (current page.south west) rectangle (current page.north east);
  \node (a) [
    program,
    anchor=north west,
    xshift=0.4cm,
    yshift=-1.4cm]
    at (current page.north west)
  {
    \adjustbox{scale=0.5}{
      \begin{minipage}[c]{0.68\textwidth}
        \begin{lstlisting}
let rec eval$^o$ fm s r =
  ...
  fm === conj x y & and$^o$ a b r &
  eval$^o$ x s a & eval$^o$ y s b |
  ...
        \end{lstlisting}
        \begin{lstlisting}
let and$^o$ x y r =
  ocanren {
    fresh xy in
      (nand$^o$ x y xy & nand$^o$ xy xy r) }

let rec nand$^o$ x y r =
  ocanren {
    (x === true & y === true & r === false) |
    (x === true & y === false & r === true) |
    (x === false & y === true & r === true) |
    (x === false & y === false & r === true) }
\end{lstlisting}
      \end{minipage}
    }
  };

  \node (b) [goal,anchor=north east] at (a.south east)
  {
    \adjustbox{scale=0.5}{
      \begin{minipage}[c]{0.24\textwidth}
        \begin{lstlisting}
eval$^o$ fm s true
        \end{lstlisting}
      \end{minipage}
    }
  };

  \pause

  \node (d) [
    transparent,
    anchor=north east,
    xshift=-0.4cm,
    yshift=-1.4cm]
    at (current page.north east)
  {
    \begin{tikzpicture}[
  processTree,
  anchor=center]
  \node (root) {\rel{eval}{$fm \ s \ true$}}
    child { node {$fm \equiv conj \ x \ y $ \and \rel{and}{$a \ b \ true$} \and \rel{eval}{$x \ s \ a$} \and \rel{eval}{$y \ s \ b$}}
      child { node {
        \rel{and}{$a \ b \ true$} \and
        \rel{eval}{$x \ s \ a$} \and
        \rel{eval}{$y \ s \ b$} \\
        \subst{$fm \to conj \ x \ y$}}
    }
    }
    ;

  \node[left=4em of root, yshift=-0.5cm] (lookup) {...};
  \draw [->] (root.west) to [out=-170,in=10] (lookup.east);

  \node[right=4em of root, yshift=-0.5cm] (lookup) {...};
  \draw [->] (root.east) to [in=170,out=-10] (lookup.west);
\end{tikzpicture}
  };

  \node (e) [
    transparent,
    anchor=north,
    yshift=-1cm]
    at (d.south)
  {
    \begin{tikzpicture}[
  processTree,
  anchor=center]
  \node {\rel{and}{$x \ y \ true$}}
    child { node {\rel{nand}{x \ y \ xy} \and \rel{nand}{$xy \ xy \ true$}}
    };
\end{tikzpicture}
  };

  \node (g) [
    transparent,
    anchor=north,
    yshift=-1cm]
    at (e.south)
  {
    \begin{tikzpicture}[
  processTree,
  anchor=center,
  sibling distance=4.5em]
  \node {\rel{nand}{$xy \ xy \ true$}}
    child { node {fail}}
    child { node {fail}}
    child { node {fail}}
    child { node {\subst{$xy \to false$}}}
    ;
\end{tikzpicture}
  };


  \node (exmpl) [
    draw=black,
    rectangle,
    semithick,
    rounded corners=2pt,
    align=center,
    anchor=south west,
    xshift=0.4cm,
    yshift=1.5cm,
    scale=0.6]
    at (current page.south west)
    {
      \rel{and}{$a \ b \ true$} \and
      \rel{eval}{$x \ s \ a$} \and
      \rel{eval}{$y \ s \ b$} \\
      \subst{$fm \to conj \ x \ y$}
    };

  \node (tip1) [transparent,anchor=south west] at ($(exmpl.north west)+(0.5,0.3)$)
  { \scriptsize
    goal
  };

  \node (tip2) [transparent,anchor=north east] at ($(exmpl.south east)+(0,-0.5)$)
  { \scriptsize
    substitution
  };

  \draw[densely dotted,->]
    (tip1.east) to
    [out=15,in=90]
    ($(exmpl.north)$);

  \draw[densely dotted,->]
    (tip2.west) to
    [out=165,in=-90]
    (exmpl.south);

    \onslide<1->
\end{tikzpicture}
  \end{center}
\end{frame}

\begin{frame}[fragile]
  \frametitle{Partial Deduction}

\begin{center}
  \begin{tikzpicture}[]
  \node (a) [
    program,
    anchor=north west,
    scale=0.6
  ]
  at (current page.north west)
  {
    \begin{lstlisting}
let double_append$^o$ x y z r =
  ocanren {
    fresh t in
      append$^o$ x y t &
      append$^o$ t z r}
\end{lstlisting}

  };

  \node [
    transparent,
    anchor=north west,
    xshift=-2cm
  ]
  at (a.south east)
    {
      \begin{minipage}[c]{\textwidth}
        \begin{tikzpicture}[
  processTree,
  level 3/.style={sibling distance=23em},
  level 4/.style={sibling distance=10em}]
  \node {\rel{double\_append}{$x \ y \ z \ r$}}
    child { node {\rel{append}{$x \ y \ t$} \and \rel{append}{$t \ z \ r$}}
      child { node[diamond] {\and}
        child { node (1) {\rel{append}{$x \ y \ t$} }
          child { node {\subst{$x \to [], y \to t$}} }
          child { node (12) {\rel{appendo}{$x' \ y \ t'$} \\ \subst{$x \to h :: x', t \to h :: t'$}}}}
        child { node (2) {\rel{appendo}{$t \ z \ r$} }
          child { node (21) {\subst{$t \to [], z \to r$}} }
          child { node (22) {\rel{appendo}{$t' \ y \ r'$} \\ \subst{$t \to h :: t', r \to h :: r'$}}}}}};
  \draw [dashed,<-] (1.east) to [out=10,in=30] (12.east);
  \draw [dashed,<-] (2.east) to [out=10,in=30] (22.east);
  \draw [red,<->] (12.south) to [out=300,in=240] (21.south);
\end{tikzpicture}
      \end{minipage}
    };


\end{tikzpicture}
\end{center}

\end{frame}

\begin{frame}[fragile]
  \frametitle{Conjunctive Partial Deduction}

\begin{center}
  \begin{tikzpicture}[]
  \node (a) [
    program,
    anchor=north west,
    scale=0.6
  ]
  at (current page.north west)
  {
    \begin{lstlisting}
let double_append$^o$ x y z r =
  ocanren {
    fresh t in
      append$^o$ x y t &
      append$^o$ t z r}
\end{lstlisting}

  };

  \node [
    program,
    anchor=south west,
    scale=0.6
  ]
  at ($(current page.south west)+(0,2)$)
  {
    \begin{lstlisting}
let double_append$^o$ x y z r =
  ocanren {
    (x === [] & append$^o$ y z r) |
    (fresh h x' r' in
      x === h :: x' &
      double_append$^o$ x' y z r' &
      r === h :: r')}
\end{lstlisting}

  };

  \node [
    transparent,
    fill=none,
    anchor=north west,
    xshift=-1cm,
    yshift=0.8cm
  ]
  at (a.south east)
    {
      \begin{minipage}[c]{0.8\textwidth}
        \begin{tikzpicture}[
  processTree,
  level 3/.style={sibling distance=20em},
  level 4/.style={sibling distance=10em},
  anchor=center]
  \node {\rel{double\_append}{$x \ y \ z \ r$}}
    child { node (0) {
      \underline{\rel{append}{$x \ y \ t$}} \and \rel{append}{$t \ z \ r$}}
      child { node[diamond] {\and}
        child { node (1) {
          \underline{\rel{append}{$t \ z \ r$}} \\
          \subst{$x \to [], y \to t$}}
          child { node {
            \subst{$t \to [], z \to r$}} }
          child { node (12) {
            \rel{append}{$t' \ z \ r'$} \\
            \subst{$t \to h :: t', r \to h :: r'$}}}}
        child { node (2) {
          \rel{append}{$x' \ y \ t'$} \and \ \underline{\rel{append}{$(h :: t') \ z \ r$}} \\
          \subst{$x \to h :: x', t \to h :: t'$}}
          child { node (22) {
            \rel{append}{$x' \ y \ t'$} \and \rel{append}{$t' \ z \ r'$} \\
            \subst{$r \to h :: r'$}}}}}};
  \draw [dashed,<-] (1.east) to [out=10,in=30] (12.east);
  \draw [dashed,<-] (0.east) .. controls +(0:8) and +(2:2) .. (22.east);

  \draw [densely dotted,->] ($(0.west)-(2,1.2)$) node[below,draw=none] {\Large call to unfold} to [out=45,in=180] (0.west);
\end{tikzpicture}
      \end{minipage}
    };


\end{tikzpicture}
\end{center}
\end{frame}


\begin{frame}[fragile]
  \frametitle{CPD: Split is Necessary}

\begin{tikzpicture}[
  remember picture,
  overlay
]
 \node (a) [
   program,
   anchor=north west,
   xshift=0.4cm,
   yshift=-1.4cm
 ]
 at (current page.north west)
 {
   \adjustbox{scale=0.6}
   {
     \begin{minipage}[c]{0.4\textwidth}
       \begin{lstlisting}
  let rec reverse$^o$ xs sx =
    ocanren {
      (xs === [] & sx === []) |
      (fresh h t t' in
        xs === h :: t &
        reverse$^o$ t t' &
        append$^o$ t' [h] sx}
  \end{lstlisting}
     \end{minipage}
   }};

 \node [
     goal,
     anchor=north east,
   ]
   at (a.south east)
   {
     \adjustbox{scale=0.6}
     {
       \begin{minipage}[c]{0.25\textwidth}
         \begin{lstlisting}
reverse$^o$ xs sx
         \end{lstlisting}
       \end{minipage}
     }};

\node [
  transparent,
  anchor=south west,
  yshift=0.5cm,
  xshift=0.4cm
  ]
  at (current page.south west)
  {
      \begin{tikzpicture}[
  processTree,
  anchor=center,
  level distance=5em,
  sibling distance=10em]
  \node (0) {
    \underline{\rel{reverse}{$xs \ sx$}}}
    child { node[diamond] {\and}
      child { node (1) {
        \rel{reverse}{$t \ t'$} \\
        \subst{$xs \to h :: t$}}
        }
      child { node {
        \rel{append}{$t' \ [h] \ xs$} \\
        \subst{$xs \to h :: t$}}
        }
      };
  \draw[dashed,->] (1.west) to [out=135,in=-155] (0.west);
\end{tikzpicture}
  };

 \node [
   transparent,
   anchor=south east,
   yshift=2.5cm,
   xshift=-0.4cm
 ]
 at (current page.south east)
 {
     \begin{tikzpicture}[processTree]
  \node {
    \underline{\rel{reverse}{$xs \ sx$}}}
    child { node {
      \underline{\rel{reverse}{$t \ t'$}} \and \rel{append}{$t' \ [h] \ sx$} \\
      \subst{$xs \to h :: t$}}
      child { node {
        \underline{\rel{reverse}{$s \ s'$}} \and \rel{append}{$s' \ [h'] \ t'$} \and \rel{append}{$t' \ [h] \ sx$} \\
        \subst{$t \to h' :: s$}}
        child { node[draw=none,fill=none ] {...}}}};
\end{tikzpicture}
 };
\end{tikzpicture}

\end{frame}

\begin{frame}[fragile]
  \frametitle{Decisions in Partial Deduction}
\begin{itemize}
  \item What to unfold: which calls, how many calls?
  \begin{itemize}
    \item CPD: the leftmost call, which does not have a predecessor \emph{embedded} into it
  \end{itemize}
  \item How to unfold: to what depth a call should be unfolded?
  \begin{itemize}
    \item CPD: unfold once
  \end{itemize}
  \item When to stop driving?
  \begin{itemize}
    \item When a goal is an instance of some goal in the process tree
  \end{itemize}
  \item When to split?
  \begin{itemize}
    \item When there is a predecessor embedded into the goal
  \end{itemize}
\end{itemize}
\end{frame}

\begin{frame}[fragile]
  \frametitle{Evaluator of Logic Formulas: Unfolding Step 1}

\begin{tikzpicture}[
  remember picture,
  overlay
]
  \node (a) [
    program,
    anchor=north west,
    xshift=0.4cm,
    yshift=-1.4cm
  ]
  at (current page.north west)
  {
    \adjustbox{scale=0.6}
    {
      \begin{minipage}[c]{\textwidth}
        \begin{lstlisting}
let rec eval$^o$ fm s r =
  ocanren { fresh v x y a b in
    (fm === var v & lookup$^o$ v s r) |
    (fm === neg x & eval$^o$ x s a & not$^o$ a r) |
    (fm === conj x y & eval$^o$ x s a & eval$^o$ y s b & and$^o$ a b r) |
    (fm === disj x y & eval$^o$ x s a & eval$^o$ y s b & oro$^o$ a b r) }
  \end{lstlisting}
      \end{minipage}
    }};

  \node [
      goal,
      anchor=north east,
    ]
    at (a.south east)
    {
      \adjustbox{scale=0.6}
      {
        \begin{minipage}[c]{0.25\textwidth}
          \begin{lstlisting}
eval$^o$ fm s true
          \end{lstlisting}
        \end{minipage}
      }};

  \node [
    transparent,
    anchor=south,
    yshift=1cm,
  ]
  at (current page.south)
  {
      \begin{tikzpicture}[
  processTree,
  sibling distance=10em,
  level distance=8em]
  \node {
    \underline{\rel{eval}{$fm \ s \ true$}}}
    child { node {
      \rel{lookup}{$x \ s \ true$} \\
      \subst{$fm \to var \ v$}}}
    child { node {
      \rel{eval}{$x \ s \ a$} \and \\
      \rel{not}{$a \ true$} \\
      \subst{$fm \to neg \ x$}}}
    child { node {
      \rel{eval}{$x \ s \ a$} \and \\
      \rel{eval}{$y \ s \ b$} \and \\
      \rel{or}{a \ b \ true} \\
      \subst{$fm \to disj \ x \ y$}}}
    child { node {
      \rel{eval}{$x \ s \ a$} \and \\
      \rel{eval}{$y \ s \ b$} \and \\
      \rel{and}{a \ b \ true} \\
      \subst{$fm \to conj \ x \ y$}}}
  ;
\end{tikzpicture}
  };
\end{tikzpicture}

\end{frame}


\begin{frame}[fragile]
  \frametitle{Evaluator of Logic Formulas: Unfolding Step 2}

\begin{center}
  \begin{tikzpicture}[
  processTree,
  sibling distance=10em,
  level distance=8em]
  \node {
    \underline{\rel{eval}{$fm \ s \ true$}}}
    child { node[draw=none, fill=none] {...}}
    child { node {
      \underline{\rel{eval}{$x \ s \ a$}} \and \\
      \rel{not}{$a \ true$} \\
      \subst{$fm \to neg \ x$}}
      child { node {
        \rel{lookup}{$v' \ s \ a$} \\
        \rel{not}{$a \ true$} \\
        \subst{$x \to var \ v'$}}}
      child { node {
        \rel{eval}{$x' \ s \ a'$} \and \\
        \rel{not}{$a' a$} \\
        \rel{not}{$a \ true$} \\
        \subst{$x \to neg \ x'$}}}
      child { node {
        \rel{eval}{$x' \ s \ a'$} \and \\
        \rel{eval}{$y' \ s \ b'$} \and \\
        \rel{or}{$a' \ b' \ a$} \\
        \rel{not}{$a \ true$} \\
        \subst{$x \to disj \ x' \ y'$}}}
      child { node {
        \rel{eval}{$x' \ s \ a'$} \and \\
        \rel{eval}{$y' \ s \ b'$} \and \\
        \rel{and}{$a' \ b' \ a$} \\
        \rel{not}{$a \ true$} \\
        \subst{$x \to conj \ x' \ y'$}}}}
    child { node[draw=none, fill=none] {...}}
    child { node[draw=none, fill=none] {...}}
  ;
\end{tikzpicture}
\end{center}
\end{frame}

\begin{frame}[fragile]
  \frametitle{Unfolding of Boolean Connectives}

  \begin{center}
    \begin{tikzpicture}[
  processTree]
  \node {
    \underline{\rel{or}{$x \ y \ true$}}}
    child { node {
      \subst{$x \to true, y \to true$}}}
    child { node {
      \subst{$x \to true, y \to false$}}}
    child { node {
      \subst{$x \to false, y \to true$}}}
  ;
\end{tikzpicture}
  \end{center}

  \vspace{1cm}

  \begin{columns}
    \begin{column}{0.5\textwidth}
      \begin{center}
        \begin{tikzpicture}[
  processTree]
  \node {
    \underline{\rel{not}{$x \ true$}}}
    child { node {
      \subst{$x \to false$}}}
  ;
\end{tikzpicture}
      \end{center}
    \end{column}
    \begin{column}{0.5\textwidth}
      \begin{center}
        \begin{tikzpicture}[
  processTree]
  \node {
    \underline{\rel{and}{$x \ y \ true$}}}
    child { node {
      \subst{$x \to true, y \to true$}}}
  ;
\end{tikzpicture}
      \end{center}
    \end{column}
  \end{columns}
\end{frame}


\begin{frame}[fragile]
  \frametitle{Unfolding Boolean Connectives First}

\begin{center}
  \begin{tikzpicture}[
  processTree,
  sibling distance=10em,
  level distance=8em]
  \node {
    \underline{\rel{eval}{$fm \ s \ true$}}}
    child { node[draw=none, fill=none] {...}}
    child { node {
      \rel{eval}{$x \ s \ a$} \and \\
      \underline{\rel{not}{$a \ true$}} \\
      \subst{$fm \to neg \ x$}}
      child { node {
        \rel{eval}{$x \ s \ false$} \\
        \subst{$a \to false$}}}}
    child { node[draw=none, fill=none] {...}}
    child { node[draw=none, fill=none] {...}}
  ;
\end{tikzpicture}
\end{center}
\end{frame}

\begin{frame}[fragile]
  \frametitle{Evaluator of Logic Formulas: Conservative PD}

\begin{center}
  \begin{tikzpicture}[
  processTree,
  level 1/.style={sibling distance=21em},
  level 2/.style={sibling distance=10em},
  level 3/.style={level distance=7em},
  level 4/.style={sibling distance=2em, level distance=5em},
  level distance=8em]
  \node (root) {
    \underline{\rel{eval}{$fm \ s \ true$}}}
    child { node {
      \rel{eval}{$x \ s \ a$} \and \\
      \underline{\rel{not}{$a \ true$}} \\
      \subst{$fm \to neg \ x$}}
      child { node {
        \rel{eval}{$x \ s \ false$} \\
        \subst{$a \to false$}}}}
    child { node {
      \rel{eval}{$x \ s \ a$} \and \\
      \rel{eval}{$y \ s \ b$} \and \\
      \underline{\rel{or}{a \ b \ true}} \\
      \subst{$fm \to disj \ x \ y$}}
      child { node {
        \rel{eval}{$x \ s \ true$} \and \\
        \rel{eval}{$y \ s \ true$} \\
        \subst{$a \to true, b \to true$}}}
      child { node {
        \rel{eval}{$x \ s \ true$} \and \\
        \rel{eval}{$y \ s \ false$} \\
        \subst{$a \to true, b \to false$}}
        child { node (rename) {
          \rel{eval}{$x \ s \ true$}}}
        child { node {
          \rel{eval}{$y \ s \ false$}}
          child { node[draw=none, fill=none] {...}}
          child { node[draw=none, fill=none] {...}}
          child { node[draw=none, fill=none] {...}}
          child { node[draw=none, fill=none] {...}}}}
      child { node {
        \rel{eval}{$x \ s \ false$} \and \\
        \rel{eval}{$y \ s \ true$} \\
        \subst{$a \to false, b \to true$}}}}
    child { node {
      \rel{eval}{$x \ s \ a$} \and \\
      \rel{eval}{$y \ s \ b$} \and \\
      \underline{\rel{and}{a \ b \ true}} \\
      \subst{$fm \to conj \ x \ y$}}
      child { node {
        \rel{eval}{$x \ s \ true$} \and \\
        \rel{eval}{$y \ s \ true$} \\
        \subst{$a \to true, b \to true$}}}}
  ;
  \node[left=8em of root] (lookup) {
    \rel{lookup}{$x \ s \ true$} \\
    \subst{$fm \to var \ v$}
  };
  \draw [->] (root.west) to [out=180,in=0] (lookup.east);
  \pause
  \draw [dashed,<-] ($(root.south west)-(-1,0)$) .. controls +(-7,-4) and +(-4,0) .. (rename.west);
\end{tikzpicture}
\end{center}
\end{frame}

\begin{frame}[fragile]
  \frametitle{Split: Which Way is the Right Way?}


\begin{tikzpicture}[
  remember picture,overlay]
  \draw[thick] (current page.south west) rectangle (current page.north east);
  \node (a) [
    transparent,
    anchor=north west,
    xshift=1.4cm,
    yshift=-1.8cm]
    at (current page.north west)
  {
    \begin{tikzpicture}[processTree,sibling distance=10em,anchor=center]
      \node {\rel{f}{$x \ y$} $\&$ \rel{g}{$x \ z$} $\&$ \rel{h}{$y \ z$}}
          child { node [diamond] {\and}
            child { node {\rel{f}{$x \ y$}}}
            child { node {\rel{g}{$x \ z$} $\&$ \rel{h}{$y \ z$}}}};
    \end{tikzpicture}
  };

  \node (d) [
    transparent,
    anchor=north east,
    xshift=-1.4cm,
    yshift=-1.8cm]
    at (current page.north east)
  {
    \begin{tikzpicture}[processTree,sibling distance=10em,anchor=center]
      \node {\rel{f}{$x \ y$} $\&$ \rel{g}{$x \ z$} $\&$ \rel{h}{$y \ z$}}
        child { node [diamond] {\and}
          child { node {\rel{f}{$x \ y$} $\&$ \rel{g}{$x \ z$}}}
          child { node {\rel{h}{$y \ z$}}}};
    \end{tikzpicture}
  };

  \draw[thick] (current page.south west) rectangle (current page.north east);

  \node (exmpl) [
    transparent,
    align=center,
    anchor=south east,
    xshift=-1.4cm,
    yshift=1.3cm]
    at (current page.south east)
    {
      \begin{tikzpicture}[processTree,sibling distance=5em,anchor=center]
        \node {\rel{f}{$x \ y$} $\&$ \rel{g}{$x \ z$} $\&$ \rel{h}{$y \ z$}}
          child { node [diamond] {\and}
            child { node {\rel{f}{$x \ y$}}}
            child { node {\rel{g}{$x \ z$}}}
            child { node {\rel{h}{$y \ z$}}}};
      \end{tikzpicture}
    };

  \node (c) [
    transparent,
    anchor=south west,
    xshift=1.4cm,
    yshift=1.3cm]
    at (current page.south west)
  {
    \begin{tikzpicture}[processTree,sibling distance=10em,anchor=center]
      \node {\rel{f}{$x \ y$} $\&$ \rel{g}{$x \ z$} $\&$ \rel{h}{$y \ z$}}
        child { node [diamond] {\and}
          child { node {\rel{f}{$x \ y$} $\&$ \rel{h}{$y \ z$}}}
          child { node {\rel{g}{$x \ z$}}}};
    \end{tikzpicture}
  };

\end{tikzpicture}
\end{frame}

\begin{frame}[fragile]
  \frametitle{Conservative Partial Deduction}
\begin{itemize}
  \item Split conjunction into individual calls
  \item Unfold each call in isolation
  \item Unfold until embedding is encountered
  \item Find a call which narrows the search state (less-branching heuristics)
  \item Join the result of unfolding the selected call with the other calls not unfolded
  \item Continue driving the constucted conjunction
\end{itemize}

\end{frame}

\begin{frame}[fragile]
  \frametitle{Less-branching Heuristics}

  \begin{center}
    Less-branching heuristics is used to select a call to unfold

    \vspace{0.5cm}

    If the call has less branches in the process tree than the relation can possible have, unfold the call
  \end{center}

\vspace{0.5cm}


  \begin{columns}
    \begin{column}[]{0.65\textwidth}
      \begin{center}
        \begin{tikzpicture}[
  processTree,
  sibling distance=12em,
  anchor=center]
  \node (1) {\rel{append}{$x \ y \ r$}}
      child { node {\subst{$x \to [], y \to r$}}}
      child { node (2) {\rel{append}{$x' \ y \ r'$ \\ \subst{$x \to h :: x', r \to h :: r'$}}}
    };
    \draw [dashed,<-] (1.east) to [out=10,in=30] (2.east);

\end{tikzpicture}
      \end{center}
    \end{column}
    \begin{column}[]{0.35\textwidth}
      \begin{center}
        \begin{tikzpicture}[processTree]
          \node {\rel{append}{$(h::x') \ y \ r$}}
              child { node {\rel{append}{$x' \ y \ r'$ \\ \subst{$r \to h :: r'$}}}
            };
        \end{tikzpicture}
      \end{center}
    \end{column}
  \end{columns}
\end{frame}

\begin{frame}[fragile]
  \frametitle{Evaluation}
We implemented the Conservative Partial Deduction and compared it with CPD for \mk and CPD with branching heuristics on the following relations

\begin{itemize}
  \item Two implementations of an evaluator of logic formulas
  \item A program to compute a unifier of two terms
  \item A program to search for paths of a specific length in a graph
\end{itemize}
\end{frame}

\begin{frame}[fragile]
  \frametitle{Evaluator of Logic Formulas}
    \begin{center}
      \adjustbox{scale=0.9}
      {
        \begin{minipage}[c]{\textwidth}
          \begin{lstlisting}
let rec eval$^o$ fm s r =
  ocanren { fresh v x y a b in
    (fm === var v & lookup$^o$ v s r) |
    (fm === neg x & eval$^o$ x s a & not$^o$ a r) |
    (fm === conj x y & eval$^o$ x s a & eval$^o$ y s b & and$^o$ a b r) |
    (fm === disj x y & eval$^o$ x s a & eval$^o$ y s b & oro$^o$ a b r) }
  \end{lstlisting}
        \end{minipage}
      }
    \end{center}
\end{frame}

\begin{frame}[fragile]
  \frametitle{Evaluator of Logic Formulas: Order of Calls}
  \begin{tikzpicture}[remember picture, overlay]

    \node (a) [
      transparent,
      anchor=north west,
      xshift=0.4cm,
      yshift=-1.8cm
    ]
    at (current page.north west)
    {
      \adjustbox{scale=0.8}
      {
        \begin{minipage}[c]{0.9\textwidth}
          \begin{lstlisting}
let rec eval$^o$ fm s r =
  ocanren { fresh v x y a b in
    (fm === var v & lookup$^o$ v s r) |
    (fm === neg x & not$^o$ a r & eval$^o$ x s a) |
    (fm === conj x y & and$^o$ a b r & eval$^o$ x s a & eval$^o$ y s b) |
    (fm === disj x y & oro$^o$ a b r & eval$^o$ x s a & eval$^o$ y s b) }
  \end{lstlisting}
        \end{minipage}
      }
    };

    \node (b) [
      transparent,
      anchor=south west]
      at (a.north west)
    {\footnotesize
        boolean connective first
    };

    \pause

    \node (c) [
      transparent,
      anchor=south east,
      xshift=-1.3cm,
      yshift=0.5cm
    ]
    at (current page.south east)
    {
      \adjustbox{scale=0.8}
      {
        \begin{minipage}[c]{0.9\textwidth}
          \begin{lstlisting}
let rec eval$^o$ fm s r =
  ocanren { fresh v x y a b in
    (fm === var v & lookup$^o$ v s r) |
    (fm === neg x & eval$^o$ x s a & not$^o$ a r) |
    (fm === conj x y & eval$^o$ x s a & eval$^o$ y s b & and$^o$ a b r) |
    (fm === disj x y & eval$^o$ x s a & eval$^o$ y s b & oro$^o$ a b r) }
  \end{lstlisting}
        \end{minipage}
      }
    };

    \node (d) [
      transparent,
      anchor=south west]
      at (c.north west)
    {\footnotesize
      boolean connective last
    };
  \end{tikzpicture}

\end{frame}

\begin{frame}[fragile]
  \frametitle{Evaluator of Logic Formulas: Compexity of Relations}


  \begin{tikzpicture}[remember picture, overlay]

    \node (a) [
      transparent,
      anchor=north west,
      xshift=0.4cm,
      yshift=-1.8cm
    ]
    at (current page.north west)
    {
      \adjustbox{scale=0.7}
      {
        \begin{minipage}[c]{0.8\textwidth}
          \begin{lstlisting}
let rec or$^o$ x y r =
  ocanren {
    (x === true & y === true & r === true) |
    (x === true & y === false & r === true) |
    (x === false & y === true & r === true) |
    (x === false & y === false & r === false) }
\end{lstlisting}
        \end{minipage}
      }
    };

    \node (b) [
      transparent,
      anchor=south]
      at (a.north)
    {\footnotesize
        table-based implementation
    };

    \pause

    \node (c) [
      transparent,
      anchor=south east,
      xshift=-0.4cm,
      yshift=0.5cm
    ]
    at (current page.south east)
    {
      \adjustbox{scale=0.7}
      {
        \begin{minipage}[c]{0.8\textwidth}
          \begin{lstlisting}
let or$^o$ x y r =
  ocanren {
    fresh a b in
      (nand$^o$ x x a & nand$^o$ y y b & nand$^o$ a b r)
  }

let rec nand$^o$ x y r =
  ocanren {
    (x === true & y === true & r === false) |
    (x === true & y === false & r === true) |
    (x === false & y === true & r === true) |
    (x === false & y === false & r === false) }
\end{lstlisting}
        \end{minipage}
      }
    };

    \node (d) [
      transparent,
      anchor=south]
      at (c.north)
    {\footnotesize
        implementation via nand$^o$
    };
  \end{tikzpicture}

\end{frame}

\begin{frame}[fragile]
  \frametitle{Evaluator of Logic Formulas: Evaluation}
Implementations:

\begin{itemize}
  \item \emph{last}: boolean connectives last, implemented via \lstinline{nand$^o$}
  \item \emph{plain}: boolean connectives first, straightforward implementation
\end{itemize}


\begin{table}
  \centering
  \begin{tabular}{c||c|c}
                   & last  & plain  \\ \hline\hline
  Original         & 1.06s & 1.84s  \\ \hline
  CPD              & ---   & 1.13s  \\ \hline
  ConsPD           & 0.93s & 0.99s  \\ \hline
  Branching        & 3.11s & 7.53s  \\ \hline
  \end{tabular}

  \caption{Evaluation results}
\end{table}
\end{frame}

\begin{frame}[fragile]
  \frametitle{Unification}
Relation to find a unifier of two terms

\vspace{0.5cm}

Query: unification of terms $f (X, X, g(Z,t))$ and $f (g(p,L),Y,Y)$
\end{frame}

\begin{frame}[fragile]
  \frametitle{Path Search}
Relation to search for paths in a graph

\vspace{0.5cm}

Query: find 5 paths in a graph with 20 vertices and 30 edges

\end{frame}

\begin{frame}[fragile]
  \frametitle{Evaluation Results}

  \begin{table}
    \centering
    \begin{tabular}{c||c|c||c||c}
                     & last  & plain & unify  & isPath \\ \hline\hline
    Original         & 1.06s & 1.84s & ---    & ---    \\ \hline
    CPD              & ---   & 1.13s & 14.12s & 3.62s  \\ \hline
    ConsPD           & 0.93s & 0.99s & 0.96s  & 2.51s  \\ \hline
    Branching        & 3.11s & 7.53s & 3.53s  & 0.54s  \\ \hline
    \end{tabular}

    \caption{Evaluation results}
  \end{table}

\end{frame}

\begin{frame}[fragile]
  \frametitle{Conclusion}
  \begin{itemize}
    \item We developped and implemented Conservative Partial Deduction
    \begin{itemize}
      \item Less-branching heuristics
    \end{itemize}
    \item Evaluation shows some improvement, but not for every query
    \item Future work:
    \begin{itemize}
      \item Develop models to predict execution time
      \item Develop specialization which is more predictable, stable and well-behaved
    \end{itemize}
  \end{itemize}
\end{frame}


\end{document}
