\section{Conjunctive Partial Deduction}

Partial deduction is a specialization technique for logic languages.

Conjunctive partial deduction is a modification of it, in which conjunctions of atoms are considered as a whole.

An adaptation of conjunctive partial deduction to \mk{} was described in an earlier paper.


Conjunctive partial deduction has its problems.
It is designed for \pro{} which fixes the order in which atoms are evaluated.
While driving, CPD consideres atoms from left to right which leads to the following problem.
If an atom which can restrict the answer set for other atoms is last in the conjunction, then before it gets unfolded, CPD would unfold all the others which are calls for free variables.
This leads to over-unfolding.

\mk{} is a language in which the set of answers does not depend on the order of calls within conjunction.
The order only affects the running time.
It would be lovely, if the specialization technique for \mk{} chooses the best order of conjuncts.
By blindly implementing CPD, we fail to do so.
This is why we created a novel specialization approach which we called non-conjunctive partial deduction.